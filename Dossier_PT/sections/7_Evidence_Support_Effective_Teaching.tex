\newpage
\sectiontitledouble{7.) Evidence in Support of}{Effective Teaching}
\fakesection{Evidence in Support of Effective Teaching}
\newpage
%For the purposes of promotion and tenure, “Effective Teaching” is defined as a demonstration of reflective teaching practices that illustrate the scope and quality of the professor’s teaching. To this end, the promotion and tenure dossier should include a structured reflection of selected information on teaching activities, provide strong evidence of teaching performance, and demonstrate approaches to evaluating teaching for improvement. The emphasis of the dossier should be placed on the strategies that the candidate employs in teaching and an assessment of their effectiveness. Student learning is a combination of both the teacher’s strategies and the students’ effort to process this information. Therefore, DFW rates, which emphasize the student’s effort, should not be taken solely into consideration for the purposes of evaluation of the candidate’s teaching effectiveness.
%At minimum, the dossier should include:
%(A). Teaching Philosophy. Provide a reflective essay about your teaching beliefs and practices. Some aspects you may wish to present include your conception of how learning occurs, description and evidence of how your teaching facilitates student learning, the goals you have for yourself and your students, description of what for you constitutes student learning and the techniques, activities, and methods you use to achieve this goal. Generic philosophical statements about teaching do not fulfill this criterion.
%(B). Evidence of Teaching Methods. Include sample syllabi, course materials, and/or assignments. [Photocopies of up to 10 pages may be included here.]
%(C). Evaluations of Teaching Effectiveness by Students. Please include printouts of course and teacher evaluations. Include here a brief narrative of your interpretation of these evaluations and what they say about you as a teacher. If applicable, include also an explanation of the cause(s) for missing evaluations.

\subsection{Teaching Philosophy}
\label{Teaching_Philosophy}
\blindtext



\subsection{Evidence of Teaching Methods}

In the following, a sample syllabus, slides, and weekly assignment for the {\it COURSE NAME} I taught in SEMESTER YEAR are enclosed.

\newpage

\includepdf[scale=0.9,pages=1-,pagecommand={}]{Files/7_Evidence_in_Support_of_Effective_Teaching/Syllabus.pdf}

\includepdf[scale=0.9,pages=1-,pagecommand={}]{Files/7_Evidence_in_Support_of_Effective_Teaching/Assignment.pdf}

\includepdf[scale=0.9,pages=1-,pagecommand={},angle=90]{Files/7_Evidence_in_Support_of_Effective_Teaching/Lecture.pdf}



\subsection{Evaluations of Teaching Effectiveness by Students}

\blindtext


\begin{table}[H]
  \centering
\begin{tcolorbox}[colback=yellow!10!white,colframe=csuOrange,title= \caption{
  \textcolor{white}{Qualitative feedback from students in the courses taught by Dr. X in the academic year 2019-2020.}}]

\taburulecolor{csuOrange}
\begin{xltabular}{\textwidth}{c | X  | l}

\toprule
Course & Students' Feedback &  \\
%%%%%%%%%%%Spring 2020%%%%%%%%%%%%%%%%
\midrule
%
\multirow{25}{*}{\rotatebox{90}{COURSE NAME}}
& \tabitem

\blindtext

& \multirow{25}{*}{\rotatebox{90}{SEMESTER YEAR}}
\\
& \tabitem
\blindtext
\\
& \tabitem
\blindtext
\\
\bottomrule
  \multicolumn{3}{r}{Continued on next page \faCaretRight\faCaretRight\faCaretRight}  \\
\bottomrule
%%%%%%%%%%%%%%%%%%%%%%%%%%%
\end{xltabular}
\end{tcolorbox}
\label{table_evaluation_2019-2020}
\end{table}

\begin{table}[H]
  \centering
\begin{tcolorbox}[colback=yellow!10!white,colframe=csuOrange,title= \caption*{
  \textcolor{white}{Table 3 -- continued from previous page}}]
\taburulecolor{csuOrange}
\begin{xltabular}{\textwidth}{c | X  | l}
\toprule
Course & Students' Feedback &  \\
%%%%%%%%%%%Fall 2019%%%%%%%%%%%%%%%%
\midrule
%
\multirow{11}{*}{\rotatebox{90}{COURSE NAME}}
& \tabitem
\blindtext
& \multirow{11}{*}{\rotatebox{90}{SEMESTER YEAR}}
\\
\midrule
%
\multirow{24}{*}{\rotatebox{90}{ANOTHER COURSE NAME}}
& \tabitem
\blindtext
& \multirow{24}{*}{\rotatebox{90}{SEMESTER YEAR}}
\\
& \tabitem
\blindtext
\\
& \tabitem
\blindtext
\\
  \bottomrule
%%%%%%%%%%%%%%%%%%%%%%%%%%%
\end{xltabular}
\end{tcolorbox}
\label{table_evaluation_2019-2020_Continued}
\end{table}


\newpage

A copy of the course and instructor evaluations for the courses Dr. X has taught from SEMESTER YEAR to SEMESTER YEAR are enclosed. The following sections did not receive the minimum number of responses required for viewing reports:
\begin{itemize}
\item SEMESTER YEAR: COURSE NAME (SECTION \#): COURSE TITLE [X out of Y Students Responded]

\item SEMESTER YEAR: COURSE NAME (SECTION \#): COURSE TITLE [X out of Y Students Responded]

\end{itemize}

\newpage

% Year 1 evaluations
\includepdf[scale=0.9,pages=1-,pagecommand={},angle=0]{Files/7_Evidence_in_Support_of_Effective_Teaching/Evaluations/Y1/Evaluation1.pdf}
%
\includepdf[scale=0.9,pages=1-,pagecommand={},angle=0]{Files/7_Evidence_in_Support_of_Effective_Teaching/Evaluations/Y1/Evaluation2.pdf}

% Year 2 evaluations
\includepdf[scale=0.9,pages=1-,pagecommand={},angle=0]{Files/7_Evidence_in_Support_of_Effective_Teaching/Evaluations/Y2/Evaluation1.pdf}
%
\includepdf[scale=0.9,pages=1-,pagecommand={},angle=0]{Files/7_Evidence_in_Support_of_Effective_Teaching/Evaluations/Y2/Evaluation2.pdf}

% Year 3 evaluations
\includepdf[scale=0.9,pages=1-,pagecommand={},angle=0]{Files/7_Evidence_in_Support_of_Effective_Teaching/Evaluations/Y3/Evaluation1.pdf}
%
\includepdf[scale=0.9,pages=1-,pagecommand={},angle=0]{Files/7_Evidence_in_Support_of_Effective_Teaching/Evaluations/Y3/Evaluation2.pdf}
